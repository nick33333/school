
 
\begin{titlepage}
    \begin{center}
        \vspace*{2cm}
        \Huge
        \textbf{Phylogenetics in a Modern-Day Pandemic} \\

        \small
        \vspace{0.5cm}
         {\Large\textbf{Nicholas Chan}}
        
        \vfill
        
\setstretch{1.2}
\section*{\fontsize{12}{15}\selectfont }     
{\normalsize
This paper covers an analysis and comparison of various tree placement programs in phylogenetics and their usage in contact tracing during the 2019-2021 COVID-19 pandemic. This paper will begin with laying out the background of phylogenetics and the idea of maximum parsimony in order to  describe the issue imposed by COVID-19, a rapidly mutating and highly transmissible virus at the center of a global pandemic. This paper will then discuss how the highest performing programs have been used for contact tracing. The intended audience of this paper are those interested in the applications of contact tracing or those who have practiced tracing lineages of organisms with bioinformatics tools. This paper is written according to the IEEE style guide.}

\Large


        
        \vspace{9cm}
        
 
        
    \end{center}
\end{titlepage}
